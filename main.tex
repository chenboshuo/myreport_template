% \documentclass[pdfCover]{myreport} % 需要pdf封面
\documentclass{myreport}
\title{测试myreport}
\author{陈伯硕}
\date{\today}

\def\MyProject{本科学年设计}

\usepackage{subfiles}
\begin{document}

\pagenumbering{roman}
% \listoftodos
% \maketitle
\subfile{sections/cover}

% Generate the Table of Contents if it's needed.
\begin{center}
	\tableofcontents
\end{center}
\newpage
\setcounter{page}{1}
\pagenumbering{arabic}

\selectlanguage{pinyin}
\begin{abstract}
  中文摘要\MyTitle
	\keywords{中文关键词1 \qquad 中文关键词2}
\end{abstract}

\selectlanguage{english}
\begin{abstract}
	Abstract in English
	\keywords{English Keywords 1; \qquad English Keywords 2;}
\end{abstract}
\selectlanguage{pinyin}

\section{section}
\subsection{subsection}
\subsubsection{subsubsection}

\cheineseCenterSection
\section{中文居中标题}
	调用命令\func{\textbackslash cheineseCenterSection}即可
	\subsection{此时二级标题不变}
		\subsubsection{三级标题也不变}

\section{中文居中标题}
		\chineseSubsection
		\subsection{加括号的中文数字做二级标题}
			调用命令\func{\textbackslash chineseSubsection}即可
			\subsubsection{三级标题不变}

		\singleArabicSubsubsection
		\subsubsection{单一数字的三级标题}
			调用命令\func{\textbackslash singleArabicSubsubsection}即可


\section{符号表}

\glsxtrnewsymbol[
	description={%
		an example symbol % 对符号的描述
	},
	unit={\si{m^2}} % 单位,不显示可以不写
]{e}{\ensuremath{\mathcal{E}}}

% 符号表
\printunsrtglossary[type=symbols,style=symbunitlong] % symbols

% 带单位的符号表
% \printunsrtglossary[type=symbols,style=symblong] % symbols without units


\section{插入代码}
\begin{codebox}
  \Procname{$bubble\_sort$($A$: the array, $n$: the length of nums)}
  \li $a = 1$
  % \li

\end{codebox}

\bibliography{reference}
\nocite{*}
\end{document}
